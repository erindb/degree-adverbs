\section{Introduction}

How do different words get their meanings?
For instance, why is an ``extremely good paper'' better than a ``quite good paper''?
The traditional answer \cite{de_saussure_nature_1916} is that different meanings have been arbitrarily and conventionally assigned to the different word forms.
This view has been challenged by a number of examples in which word meaning appears to be non-arbitrarily related to properties of the word.
In some cases, the phonetic form of a word is systematically related to its meaning, for example rounded vowels and voiced consonants tend to refer to round objects \cite{kohler_gestalt_1970, ramachandran_synaesthesiawindow_2001, holland_physiognomic_1964, davis_fitness_1961}.
In other cases, orthographic form is diagnostic of meaning, for example, speakers of Hebrew who have never seen Chinese characters are nonetheless above chance at matching them to their corresponding Hebrew words \cite{koriat_figural_1979}.
Similarly, the length of words predicts aspects of their meanings: across languages longer words refer to more complex meanings \cite{lewis_conceptual_2016}.
Open questions remain about the systematic factors that can influence meaning and the source of these effects.

In this paper, we explore adjectival intensifiers\footnote{
Intensifiers are adverbs that modify scalar adjectives so that the interpretation of the intensified adjective phrase is more extreme than the interpretation of the bare adjective phrase.
The word ``intensifier'' is often used to denote the full range of degree adverbs, be they ``amplifiers'', or ``downtoners'' \cite{quirk_comprehensive_1985}.
The ``intensifiers'' we are looking at in this paper are, according to this typology, ``amplifiers'' because they increase (rather than decrease) the threshold associated with a gradable predicate.
This typology also distinguishes between two different kinds of amplifiers: those that increase an adjective maximally (e.g. \w{completely} and \w{utterly}) and those that merely increase (e.g. \w{greatly} and \w{terribly}).
We do not make this distinction.
The word ``intensifier'' is sometimes used for a completely different linguistic phenomenon, where a reflexive is used for emphasis, e.g. ``The king himself gave the command,'' which we do not analyze in this paper. 
},
like \w{extremely} and \w{quite}, as a case study in which to empirically explore the relationship of meaning to factors like word form and distribution of usage.
Intensifiers form a good case study because they are amenable to simple quantitative measures of meaning: Many adjectives correspond to concrete numeric scales, and intensifier's strength can be measured as the numeric extent to which it shifts the interpretation of such a scalar adjective. Intesifiers are of interest because theoretical considerations, which we lay out below, suggest a relationship between intensifier meaning and their communicative cost (i.e. frequency and length).
This account of intensifier meaning adds to a growing body of literature exploring how principles of recursive, rational communication shape language interpretation (e.g. \citeNP{grice_logic_1975, frank_predicting_2012, goodman_knowledge_2013, franke_quantity_2011, russell_probabilistic_2012, kao_nonliteral_2014, bergen_pragmatic_2014}).

In the next \hyperref[sec:semantics]{section}, we discuss a minimal semantics for intensifiers, building off of previous work on scalar adjectives.
We show how pragmatic effects predict systematic variation in the meanings of intensifiers: the meanings of intensifiers are expected to be influenced by their form (in length) and their distribution (frequency) of usage.
\eb{We formalize this semantics in our \hyperref[app:model]{Appendix}, and derive the prediction that the interpreted strength of an intensified phrase should be linearly related to communicative cost (i.e. length and frequency) of that phrase.}
The impact of word length is reminiscent of the results of \citeA{lewis_conceptual_2016}, who studied noun categories.
While word frequency is known to have major effects on sentence processing (e.g. \citeNP{levy_expectation-based_2008}), the prediction that frequency should affect meaning is more surprising.

We confirm, in our first series of studies (Studies \hyperref[sec:study1a]{1a}, \hyperref[sec:study1b]{1b}, and \hyperref[sec:study2]{2}), that English intensifiers in adjective phrases are indeed interpreted as much stronger for less frequent intensifiers.
This holds in quantitative judgments of meaning and in forced comparisons, and across a number of adjectival dimensions.
With the more sensitive dependent measure of Study 2, we also find an additional effect of length above and beyond surprisal.
In our second set of studies (Studies \hyperref[sec:study3]{3} and \hyperref[sec:study4]{4}), we replicate this finding, and extend it to novel intensifiers, showing that length is a significant predictor of the strength of an intensifier's meaning even in the absence of any conventional meaning.
We conclude with a discussion of different interpretations of these phenomena and future directions.
