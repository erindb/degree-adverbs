\section{The semantics of intensifying degree adverbs \label{sec:semantics}}

Our paper focuses on intensifying degree adverbs applied to scalar adjectives.\footnote{
  Some of these intensifiers can also apply to verbal and nominal predicates, and different restrictions apply for different intensifiers, e.g. \w{I truly like carrots} is an acceptable utterance, whereas \w{I very like carrots} is not. See \citeA{bolinger_degree_1972} for a discussion.
 }
Scalar adjectives have been described as having a threshold semantics \cite{kennedy_vagueness_2007}, where, for example, \w{expensive} means ``having a price greater than $\theta$'' and $\theta$ is a semantic variable inferred from context (e.g., \$100).
Above the threshold degree $\theta$, the adjective is true of an object, and below, the adjective is false.
\citeA{lassiter_context_2013} build on the Rational Speech Acts (RSA) framework \cite{frank_predicting_2012, goodman_knowledge_2013} to give a formal, probabilistic model of how this threshold might be established by pragmatic inference that takes into account statistical background knowledge (such as the distribution of prices for objects).
We return to this model below and present a full model in the \hyperref[app:model]{Appendix}.

Previous researchers have proposed that adjective phrases modified by intensifiers have the same semantics as unmodified adjective phrases, except with new, higher thresholds \cite{kennedy_scale_2005, klein_semantics_1980, wheeler_attributives_1972}.
That is, some threshold, inferred from context, exists above which objects are \w{expensive} and below which they are not, and the intensifier \w{very} determines a new, higher threshold for the adjective phrase \w{very expensive}.
These researchers suggest that the intensified thresholds are determined by first collecting the set of objects in the comparison class for which the bare adjective is true, and then using that as the comparison class to infer a new threshold, i.e. \w{very expensive laptop} means ``expensive for an expensive laptop''.
This analysis results in the expected intensification of adjectives (``expensive for an expensive laptop'' has a higher threshold for being true than simply ``expensive for a laptop'') and is appropriately sensitive to different domains (e.g. the absolute difference in price between thresholds for \w{expensive} and \w{very expensive} is much higher in the context of ``That space station is very expensive,'' than in the context of ``That coffee is very expensive.'').
However, this proposal does not distinguish between the graded strengths of different intensifiers, for example, \w{very expensive} and \w{phenomenally expensive}.

Intuition suggests that different intensifiers do have different strengths (e.g. \w{outrageously} seems stronger than \w{quite}), and we provide further evidence of this in our studies, where participants interpret and compare different intensifiers.
It could be that the degree of strength of different intensifiers is conventionally specified by the lexicon. But the semantics must then specify how these entries affect the very flexible threshold of the relevant adjective.
In addition, the multitude of intensifiers \cite{bolinger_degree_1972} and their apparent productivity\footnote{For example, \w{altitidinously expensive} is not in common usage, but one can easily interpret \w{altitidinously} as a novel intensifier.}
suggest a more parsimonious solution would be welcome. 
That is, having a lexically determined meaning for each different intensifier might overlook the similarity among words of this class.
In the account that follows, we build minimally on existing models of adjective interpretation and rational communication to articulate a model of intensified adjective phrase interpretation.
