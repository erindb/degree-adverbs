\documentclass[10pt,letterpaper]{article}

\usepackage{cogsci}
\usepackage{pslatex}
\usepackage{apacite}

\title{Word frequency and the meanings of degree adverbs}
 
\author{{\large \bf Erin Bennett (erindb@stanford.edu)} \\
  Department of Psychology, 450 Serra Mall \\
  Stanford, CA ????? USA
  \AND {\large \bf Noah Goodman (ngoodman@stanford.edu)} \\
  Department of Psychology, 450 Serra Mall \\
  Stanford, CA ????? USA}

\begin{document}

\maketitle

\begin{abstract}
Can word frequency predict the meanings of degree adverbs?

\textbf{Keywords:} 
degree adverbs; word frequency.
\end{abstract}

\section{Introduction}

Degree adverbs are adverbs that modify the degree to which a particular adjective applies, for example ``very'', ``extremely'', ``moderately'', or ``kinda''. We will call degree adverbs that increase the extent to which an adjective applies (e.g. ``very'') \emph{intensifying} degree adverbs, and those that lessen the extent to which an adjective applies \emph{deintensifying} degree adverbs.

We hypothesize that the meaning of these adverbs is derived, at least in part, from the cost to the speaker of choosing a particular adverb over another. This cost is influenced by the frequency with which that adverb is used (frequent adverbs are less costly and infrequent adverbs are more costly), and by the length of the word (longer words or phrases are more costly than shorter ones). Therefore, by our hypothesis, degree adverbs that are short and common will modify adjective meanings only a small amount, but degree adverbs that are very long and uncommon will modify adjective meanings much more.

We test this hypothesis by eliciting interpretations of different degree adverbs when combined with the adjective ``expensive'' and comparing these interpretations to the adverbs' syllable length and corpus frequencies.

\section{Experiment 1}
  \subsection{Method}
    \subsubsection{Participants}
    We recruited 80 participants on Amazon's Mechanical Turk.
    \subsubsection{Procedure and Materials}
  \subsubsection{Results}
  \subsection{Discussion}

\section{Experiment 2}

  In an attempt to deconfound the findings of Experiment 1, we ran Experiment 2 on a new set of degree adverbs: deintensifying degree adverbs. We assumed that words like \emph{moderately} modify the meaning of the adjective they modify so that the object in question is closer to the average version of that object, i.e. that a ``moderately expensive laptop'' would mean ``closer to the average price of laptops than an expensive laptop''. Then the stronger this deintenfying degree adverb's meaning, the more average the object being described would be. Given this assumption, we could better test our hypothesis that degree adverbs' strength is derived from their frequency. While a strong intensifying degree adverb would describe unusual and therefore possibly infrequent items, a strong deintensifying adverb would describe more usual and therefore more frequent items.
  
  If infrequent degree adverbs have stronger meanings, we would expect a positive correlation between frequency and degree in the case of degree adverbs. If more frequent items are more likely to be talked about and therefore degree adverbs that describe these more frequent items are more likely to be used, then we would expect a negative correlation between frequency and degree.
  
  %We assumed that if the meaning of a deintensifying degree adverb was stronger, it would describe a more adverage, canonical object.
  
  \subsection{Method}
    \subsubsection{Participants}
    We recruited 10 participants on Amazon's Mechanical Turk.
    \subsubsection{Procedure and Materials}
  \subsection{Results}
  \subsection{Discussion}
  
  Our findings in Experiment 2 might provide evidence that our findings in Experiment 1 are due to the frequencies of the items being described rather than to degree adverbs derving their meanings from their frequencies.
  
  However, this is inconsistent with the finding that syllable length, on top of inverse frequency, is a marginally significant predictor of intensifying degree adverb strength. It may be the case that our assumption about the meanings of deintensifying degree adverbs was flawed. Possibly deintensifying degree adverbs do not merely indicate the item in question is closer to average. We therefore ran Experiment 3 where rather than using deintensifying adverbs, we used intensifying adverbs paired with ``average'', which more likely carries the meaning we assumed deintensifiers had.
  
\section{Experiment 3}

  In Experiment 3, we attempted to replicate our findings about frequency and degree on a slightly refined set of degree adverbs and nouns. At the same time, we ask the same question as Experiment 2 (Are our findings in Experiment 1 due to the fact that infrequent degree adverbs consequently have stronger meanings, or due to the fact that infrequent items are less likely to be talked about?) with less uncertainty about the meanings of the sentences we give participants. 
  
  \subsection{Method}
    \subsubsection{Participants}
    \subsubsection{Procedure and Materials}
    
     We chose our new set of adverbs to have minimal lexical ambiguity, so that the 1-gram frequencies accurately reflect the frequency which which these words are used as modifiers of degree. We avoided adverbs that are frequently used to communicate properties independent of degree: For example, we avoided \emph{crazy} because it often communicates erratic behavior or madness and \emph{super} because is often used in the context of comic books heros.
     
     We also modified our set of nouns to include \emph{sweater} and \emph{headphones} (purchases we though Mechanical Turk workers would be somewhat familiar with) and to no longer included \emph{watch}. Because of the new popularity of smartwatches, \emph{watch} is now somewhat lexically ambiguous.
     
     Items in the replication condition were presented in the same way as in Experiment 1, and items in the new condition were changed to say ``It was a(n) [adverb] average price.''
     
  \subsection{Results}
  \subsection{Discussion}

\section{Acknowledgments}

Acknowledgments.

\nocite{label}


\bibliographystyle{apacite}

\setlength{\bibleftmargin}{.125in}
\setlength{\bibindent}{-\bibleftmargin}

\bibliography{degree-adverbs}


\end{document}
